% This is "sig-alternate.tex" V2.0 May 2012
% This file should be compiled with V2.5 of "sig-alternate.cls" May 2012
%
% This example file demonstrates the use of the 'sig-alternate.cls'
% V2.5 LaTeX2e document class file. It is for those submitting
% articles to ACM Conference Proceedings WHO DO NOT WISH TO
% STRICTLY ADHERE TO THE SIGS (PUBS-BOARD-ENDORSED) STYLE.
% The 'sig-alternate.cls' file will produce a similar-looking,
% albeit, 'tighter' paper resulting in, invariably, fewer pages.
%
% ----------------------------------------------------------------------------------------------------------------
% This .tex file (and associated .cls V2.5) produces:
%       1) The Permission Statement
%       2) The Conference (location) Info information
%       3) The Copyright Line with ACM data
%       4) NO page numbers
%
% as against the acm_proc_article-sp.cls file which
% DOES NOT produce 1) thru' 3) above.
%
% Using 'sig-alternate.cls' you have control, however, from within
% the source .tex file, over both the CopyrightYear
% (defaulted to 200X) and the ACM Copyright Data
% (defaulted to X-XXXXX-XX-X/XX/XX).
% e.g.
% \CopyrightYear{2007} will cause 2007 to appear in the copyright line.
% \crdata{0-12345-67-8/90/12} will cause 0-12345-67-8/90/12 to appear in the copyright line.
%
% ---------------------------------------------------------------------------------------------------------------
% This .tex source is an example which *does* use
% the .bib file (from which the .bbl file % is produced).
% REMEMBER HOWEVER: After having produced the .bbl file,
% and prior to final submission, you *NEED* to 'insert'
% your .bbl file into your source .tex file so as to provide
% ONE 'self-contained' source file.
%
% ================= IF YOU HAVE QUESTIONS =======================
% Questions regarding the SIGS styles, SIGS policies and
% procedures, Conferences etc. should be sent to
% Adrienne Griscti (griscti@acm.org)
%
% Technical questions _only_ to
% Gerald Murray (murray@hq.acm.org)
% ===============================================================
%
% For tracking purposes - this is V2.0 - May 2012

%\documentclass{acm_proc_article-sp}
\documentclass{sig-alternate}
\usepackage{color}
\usepackage{multirow}
\usepackage{listings}
\usepackage{url}

\lstset{language=java, numbers=left, numberstyle=\tiny\color{black}, numbersep=3pt, escapeinside={\$}, basicstyle=\footnotesize\ttfamily, escapeinside={\%*}{*)}}

\newif\ifdraft
\drafttrue
%\draftfalse                                                                              
\ifdraft
\newcommand{\zhaonote}[1]{{\textcolor{cyan}    { ***Zhao:      #1 }}}
\newcommand{\note}[1]{ {\textcolor{red}    {\bf #1 }}}
\else
\newcommand{\zhaonote}[1]{}
\newcommand{\note}[1]{}
\fi


\begin{document}

\title{Snapshot: Diagnosing Data with Machine Learning Pipeline Lineage}

\numberofauthors{3} 
\author{
% 1st. author
\alignauthor Zhao~Zhang\\\
       \affaddr{AMPLab}\\
       \affaddr{University of California, Berkeley} \\
       \email{zhangzhao@berkeley.edu}    
% 4th. author       
\alignauthor Evan~Sparks\\\
       \affaddr{AMPLab}\\
       \affaddr{University of California, Berkeley}\\
       \email{sparks@berkeley.edu}       
% 6th. author
\alignauthor Michael~J.~Franklin\\
       \affaddr{AMPLab}\\
       \affaddr{University of California, Berkeley}\\
       \email{franklin@berkeley.edu}   
}

\maketitle

\begin{abstract}
Modern computing systems are often capable of collecting fine grained lineage to help
users understand the results, and possibly find data anomalies. 
We present the Snapshot lineage system as part of the KeystoneML machine learning pipeline. 

\end{abstract}

% A category with the (minimum) three required fields
\category{I}{Have}[No Idea]
%\keywords{ACM proceedings, \LaTeX, text tagging}

\section{Introduction}
Hadoop~\cite{HADOOP} 
 
\section{Background}
\label{sec:Background}

\subsection{KeystoneML}

\subsection{Use Cases}


\section{Conclusion}
\label{sec:Conclusion}


%ACKNOWLEDGMENTS are optional
\section{Acknowledgments}

This research is supported in part by NSF CISE Expeditions Award CCF-1139158, LBNL Award 7076018, and DARPA XData Award FA8750-12-2-0331, and gifts from Amazon Web Services, Google, SAP,  The Thomas and Stacey Siebel Foundation, Adatao, Adobe, Apple, Inc., Blue Goji, Bosch, C3Energy, Cisco, Cray, Cloudera, EMC, Ericsson, Facebook, Guavus, Huawei, Intel, Microsoft, NetApp, Pivotal, Samsung, Splunk, Virdata, VMware, and Yahoo!. Author FAN is supported by a National Science Foundation Graduate Research Fellowship.

This research is also supported in part by the Gordon and Betty Moore
Foundation and the Alfred P. Sloan Foundation together through the
Moore-Sloan Data Science Environment program.
%
% The following two commands are all you need in the
% initial runs of your .tex file to
% produce the bibliography for the citations in your paper.
\bibliographystyle{abbrv}
\bibliography{Lineage} % sigproc.bib is the name of the Bibliography in this case
% You must have a proper ".bib" file
%  and remember to run:
% latex bibtex latex latex
% to resolve all references
%
% ACM needs 'a single self-contained file'!
%
%APPENDICES are optional
%\balancecolumns



\balancecolumns

% That's all folks!
\end{document}
